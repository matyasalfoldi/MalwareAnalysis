\documentclass[11pt,a4paper]{article}
\begin{document}
\title{Static analysis of WannaCry}
\author{Alföldi Mátyás}
\maketitle
\tableofcontents
\pagebreak
\section*{Introduction}
\markright{}
\addcontentsline{toc}{section}{Introduction}
Sample source: https://github.com/ytisf/theZoo \newline\newline
Tools used:
\begin{itemize}
\item Ghidra
\item CFF Explorer
\item Git bash(for file, etc.)
\item VirtualBox
\item Exeinfo PE
\item strings.exe from SysinternalSuite
\item oletools2
\end{itemize}
\pagebreak
\section*{First Steps}
\markright{}
\addcontentsline{toc}{section}{First Steps}
\begin{enumerate}
\item Looking at the file in Exeinfo PE I see that a .Zip is found in section 4, and below it it also shows, that section 4, which is the .rsrc is big. This already hints that it might unpack more malware while running.
\item Opening it in CFF Explorer, and checking the Resource Editor tab shows that there is a XIA resource named 2058 that starts with the zip header.
\item Saving the resource and trying to unzip it results in a password being asked, which is most likely found in the file.
\item Using strings on the file, and looking through the output, we can see multiple interesting things:
\begin{enumerate}
\item Most likely messages designed for different languages: msg/m\_\textit{language}.wnry, also other .wnry files
\item Various exes: diskpart.exe, taskdl.exe, taskse.exe, tasksche.exe, cmd.exe /c "\%s"
\item Various dll-s that are most likely loaded with LoadLibrary: kernel32.dll, advapi32.dll, msvcp60.dll, msvcrt.dll, ws\_32.dll, oleaut32.dll, shell32.dll, user32.dll
\item Using a filter for winapi calls from here: https://resources.infosecinstitute.com/windows-functions-in-malware-analysis-cheat-sheet-part-1/ gives us some findings.
\end{enumerate}
\end{enumerate}
\pagebreak
\section*{Interesting things found with Ghidra}
\markright{}
\addcontentsline{toc}{section}{Interesting things found with Ghidra}
Note: All files were renamed, so that their extension is only .ex .\newline
Starting at the WinMain function we can see that calls GetModuleFileNameA for gettings its path, and it generates afterwards a random string.
Afterwards we can see that it behaves in different ways:
\begin{enumerate}
\item If it has /i as its argument.
\item Any other cases.
\end{enumerate}
Lets start with the first one:\newline
\begin{enumerate}
\item Creation of a hidden system directory(First successful of these):
\begin{itemize}
\item \%WinDir\% \textbackslash ProgramData
\item \%WinDir\% \textbackslash Intel
\item TMP folder
\end{itemize}
\item Using CopyFileA for naming itself as tasksche.exe
\item Creates an automatic service running tasksche.exe or just a hidden process if the first fails. (Using the random string generated at the beginning.
\end{enumerate}
Second case:
\begin{enumerate}
\item Sets the directory to where the program is running.
\item Creates the following registry key(First successful):
\begin{itemize}
\item HKLM \textbackslash Software \textbackslash WanaCrypt0r
\item HKCU \textbackslash Software \textbackslash WanaCrypt0r
\end{itemize}
\item We also set in the above mentioned key the following way:
\begin{itemize}
\item dw value
\item String type
\item With the value returned by GetCurrentDirectoryA
\item Size: Length of the value returned + 1 (for zero terminator, also regedit only shows values for strings till the first zero terminator, hiding things would be possible after the zero terminator)
\end{itemize}
\item unpacks the resources found at the start, here we see that the password is: WNcry@2ol7 (TODO: go more in depth in the unpacking algo)
\item Modifies the extracted c.wnry to contain one of 3 bitcoin addresses. More about c.wnry in its subsection.
\item Adds the hidden attribute to the current directory.
\item Grans everyone Full access to the current directory.
\item Uses LoadLibraryA to load advapi32.dll and kernel32.dll to gather func ptrs to the following:
\begin{itemize}
\item Cryptography related functions:
\begin{itemize}
\item CryptAcquireContextA
\item CryptImportKey
\item CryptDestroyKey
\item CryptEncrypt
\item CryptDecrypt
\item CryptGenKey
\end{itemize}
\item File operation functions:
\begin{itemize}
\item CreateFileW
\item WriteFile
\item ReadFile
\item MoveFile
\item MoveFileExW
\item DeleteFileW
\item CloseHandle
\end{itemize}
\end{itemize}
\item Imports a cryptographic key from an in memory key BLOB (public/private key pair)
\item Starts parsing t.wnry, which contains an AES key, that is later on decrypted with the help of the above mentioned cryptographic key
\item There is also some custom rijndael implementation, which I found out with the help of FindCrypt script.
\item TODO: continue the parsing process of t.wnry
\end{enumerate}
\pagebreak
\section*{Extracted files}
\markright{}
\addcontentsline{toc}{section}{Extracted files}
\subsection{b.wnry}
\subsection{c.wnry}
c.wnry by default contains the following:
\begin{itemize}
\item onion links
\item A link to tor
\end{itemize}
\subsection{r.wnry}
The Ransomware note.
\subsection{s.wnry}
Another zip file containing:
\begin{itemize}
\item RTF files in different languages, which might be malicious too.
\item A Basic tor (TODO: check if it is a modified version)
\end{itemize}
\subsection{t.wnry}
Contains the following:
\begin{itemize}
\item An 8 byte magic number: WANACRY!
\item A 4 byte key size: 00 00 01 00 = 256 AES has this length
\item Another 4 bytes TODO: what exactly it is
\item Another 8 bytes
\item TODO
\end{itemize}
\subsection{taskdl.exe}
\subsection{taskse.exe}
\subsection{u.wnry}
\subsection{msg/m\_\textit{language}.wnry files}
After looking at it in CFF Explorers hex view, I have seen some strange parts in it.\newline
Generating its hash with sigcheck and looking it up in virustotal showed me that my suspicion was correct.
\pagebreak
\section*{yara rules}
\markright{}
\addcontentsline{toc}{section}{yara rules}
\pagebreak
\section*{Snort rules}
\markright{}
\addcontentsline{toc}{section}{Snort rules}
\end{document}