\documentclass[11pt,a4paper]{article}
\usepackage[utf8]{inputenc} 
\begin{document}
\title{Analysing Malware Traffic}
\author{Alföldi Mátyás}
\maketitle
\tableofcontents
\pagebreak
\section*{Introduction}
\markright{}
\addcontentsline{toc}{section}{Introduction}
Source: http://www.malware-traffic-analysis.net/2020/01/30/index.html\newline
Basic info:\newline
LAN segment data:
\begin{itemize}
\item LAN segment range:  10.20.30.0/24 (10.20.30.0 through 10.20.30.255)
\item Domain:  sol-lightnet.com
\item Domain controller:  10.20.30.2 - Sol-Lightnet-DC
\item LAN segment gateway:  10.20.30.1
\item LAN segment broadcast address:  10.20.30.255
\end{itemize}
\pagebreak
\section*{Interesting Alerts}
\markright{}
\addcontentsline{toc}{section}{Interesting Alerts}
First few interesting Alerts:
\begin{itemize}
\item ET POLICY exe download via HTTP - Informational
\item ET CURRENT\_EVENTS Terse alphanumeric executable downloader high likelihood of being hostile
\item ET POLICY Binary Download Smaller than 1 MB Likely Hostile
\item ET POLICY PE EXE or DLL Windows file download HTTP
\end{itemize}
The IP related to this is:
49.51.133.162
The virustotal link shows that 2 detected files communicate with this IP.\newline
(https://www.virustotal.com/gui/ip-address/49.51.133.162/details)\newline
The abuse.ch lookup for this shows that the downloaded sv.exe is the Pony Trojan (https://urlhaus.abuse.ch/host/gengrasjeepram.com/), but looking into the virustotal link multiple AV names it Hancitor, which explains the later Hancitor related alerts.\newline
(https://www.virustotal.com/gui/file/995cbbb422634d497d65e12454cd\newline
5832cf1b4422189d9ec06efa88ed56891cda)\newline\newline
ET POLICY External IP Lookup api.ipify.org\newline
This seems mostly like this service is used by the malware, but is not necessarily malicious.\newline\newline
ETPRO TROJAN Tordal/Hancitor/Chanitor Checkin\newline
The IP address: 81.177.6.156 is also reported to have been malicious.\newline
This one is related to the Hancitor malware.\newline\newline
ETPRO TROJAN Hancitor encrypted payload Jan 17 (1)\newline
Here we can see communication with 148.66.137.40, which when looking up virustotal 10+ known malicious files communicate with it.\newline\newline
ET TROJAN VMProtect Packed Binary Inbound via HTTP - Likely Hostile\newline
49.51.133.162 also seems to be related to gengrasjeepram.com
\pagebreak
\section*{Pcap Analysis}
\markright{}
\addcontentsline{toc}{section}{Pcap Analysis}
Information about the infected host:
\begin{itemize}
\item Host: DESKTOP-4C02EMG
\item User: alejandrina.hogue
\item IP: 10.20.30.227
\item Mac: 58:94:6b:77:9b:3c
\item Windows version: win 8 most likely from the data exfiltration
\end{itemize}
Communications mentioned in the alerts:
\subsection{49.51.133.162}
The only communication with this server is during the sv.exe download.
\subsection{81.177.6.156}
Seems like the communication starts with data exfiltration.
The following info is exfiltrated:
\begin{itemize}
\item Some Guid
\item Build Version of a software
\item hostname@domain\textbackslash user
\item Windows version
\item External IP
\item Type (integer) 
\end{itemize}
What is interesting is that during the HTTP OK message from the server, something that looks like base64 is sent back.\newline
Most likely the application that communicates with this ip (probably sv.exe) can understand it.\newline
Later on it seems that more data is exfiltrated and again an encoded messages is passed back with the HTTP OK.\newline
Although it seems as if the malware exfiltrates the same data multiple times, might be a bug in it.
\subsection{148.66.137.40}
Two pictures are gathered with get requests, which result in encrypted data being returned.
\end{document}