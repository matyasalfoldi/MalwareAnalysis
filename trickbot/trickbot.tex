\documentclass[11pt,a4paper]{article}
\usepackage[utf8]{inputenc}
\begin{document}
\title{Trickbot}
\author{Alföldi Mátyás}
\maketitle
\tableofcontents
\pagebreak
\section*{Introduction}
\markright{}
\addcontentsline{toc}{section}{Introduction}
File: imgpaper.png\newline
VT: https://www.virustotal.com/gui/file/934c84524389ecfb3b1dfcb28f\newline
9697a2b52ea0ebcaa510469f0d2d9086bcc79a/detection
\newpage
\section*{Initial VT report check}
\markright{}
\addcontentsline{toc}{section}{Initial VT report check}
Sections:
\begin{itemize}
\item .text section seems to have a decently high entropy, so there will most likely be an unpacking process.
\item .rsrc section is quite big, with a close to 8 entropy, signaling that it is definitely packed.
\end{itemize}
Interesting Imports:
\begin{itemize}
\item advapi32.dll
 \begin{itemize}
 \item Various registry related api calls + Get/SetFileSecurity
 \end{itemize}
\item kernel32.dll
 \begin{itemize}
 \item Virtual* functions
 \item Resource handling functions (for the unpacking)
 \item Heap related functions
 \item File operations (like Read/Write file)
 \item Library loading/unloading
 \item SetUnhandledExceptionFilter + UnhandledExceptionFilter combo
 \item Various Get* functions for information gathering
 \end{itemize}
\item WINSPOOL.DRV (This in itself is strange)
\end{itemize}
\newpage
\section*{Static Analysis}
\markright{}
\addcontentsline{toc}{section}{Static Analysis}
\subsection{imgpaper.png}
Once entering WinMain it immediately gets the current thread with AfxGetThread, the returned value is later on used, to calculate function pointers to call(By adding x to the returned value).\newline
Afterwards it tries to Load a string resource, which doesn't seem to exist.\newline
Installs a hook regarding input into dialogbox,msgbox,etc. for the current thread.\newline
In between there is a calculated func ptr call, and it undo-s the hook and calls UnregisterClassA.\newline
Here will most likely the interesting part start, since after the WinMain there is a bunch of calculated func ptr call, and there might be some anti-debugging code.\newline
The RCDATA in the .rsrc section seems to be the cause for thehigh entropy, most likely this stores some PE file.\newline
Preparing for debugging it:\newline
As one can see, it will create a child process, so setting a bp on CreateProcessInternalW/A, VirtualAlloc, VirtualProtect, WriteProcessMemory, NtWriteVirtualMemory will most likely help us catch the probably unpacked PE file that gets inserted into some system process. (Most likely Process Hollowing will take place.)\newline
Debugging:\newline
It seems to store some system info in the first VirtualAlloc-ed location.\newline
And just as I thought the second VirtualAlloc-ed location has a PE file without the MZ magic number(Although at the first VirtualProtect time, it only has a the .text section ready).\newline
Afterwards the next VirtualAlloc-ed location has another PE in it(check if they are the same). This one is already detected on virustital by a few engines:\newline
https://www.virustotal.com/gui/file/b19fda07723593cd078e14ebec6ef535256\newline 
a44bd5377c96d044923406374d158/detection\newline
Next it halted on CreateProcessInternalW, where it tries to start wermgr.exe, and obviously it is started in suspended state, but it seems to not use WriteProcessMemory, or NtWriteVirtualMemory to do the copy over.\newline
Also it is wise to attach a debugger, to the suspended process, since the main process exits.
\end{document}